\documentclass{sig-alternate-05-2015}
\usepackage[svgnames]{xcolor}

\usepackage{listings}
\usepackage{fancyvrb}
\usepackage{balance}
\usepackage{colonequals}
\usepackage{paralist}
\usepackage[skins]{tcolorbox}
\usepackage{booktabs}
\usepackage{tabularx}
\usepackage[resetfonts]{cmap} %helps dealing with ligatures (special symbols for ff, fi etc that prevent searching)

\usepackage{mathtools}

\usepackage{xfrac}
% subcaption is a better alternative to subfigure, cf. e.g.
% http://tex.stackexchange.com/questions/13625/subcaption-vs-subfig-best-package-for-referencing-a-subfigure
%\usepackage{graphicx}
\usepackage{caption}
\DeclareCaptionType{copyrightbox}
\usepackage{subcaption}

\usepackage[linesnumbered,ruled,vlined,boxed]{algorithm2e}
\renewcommand\AlCapFnt{\normalfont}
%% more spacing
\SetAlCapSkip{0.5em}
\let\chapter\undefined % Magic trick that fixes appendix links
\newtheorem{theorem}{Theorem}[section]
\newtheorem{proposition}[theorem]{Proposition}
\newtheorem{lemma}[theorem]{Lemma}
\newtheorem{corollary}[theorem]{Corollary}
\newtheorem{observation}[theorem]{Observation}
\newtheorem{example}[theorem]{Example}
\newtheorem{definition}[theorem]{Definition}

% fancy verbatim
\fvset{frame=single,framesep=1mm,fontfamily=tt,numbers=left,framerule=.3mm,numbersep=1mm,commandchars=\\\{\}}
\definecolor{bgred}{RGB}{255,210,205}
\definecolor{bgblue}{RGB}{210,220,255}
\definecolor{bgyellow}{RGB}{240,240,190}
\definecolor{bggrey}{RGB}{223,223,225}
\definecolor{purple}{RGB}{180,0,180}

%\newcommand{\REMARK}[2]{}
\newcommand{\REMARK}[2]{\textcolor{red}{[{\bf #1: }#2]}}
\newcommand{\eyal}[1]{\REMARK{Eyal}{#1}}
\newcommand{\tova}[1]{\REMARK{Tova}{#1}}
\newcommand{\scream}[1]{\REMARK{\underline{TBD}}{#1}}

% general macros
\newcommand{\var}[1]{\operatorname{\mathit{#1}}}
\newcommand{\name}[1]{\operatorname{#1}}
\newcommand{\funct}[2]{\operatorname{#1}\!\left({#2}\right)}
\newcommand{\To}{\!\rightarrow\!}
\newcommand{\card}[1]{\left|{#1}\right|}
\newcommand{\ceil}[1]{\left\lceil{#1}\right\rceil}
\renewcommand{\epsilon}{\varepsilon}
\renewcommand{\phi}{\varphi}
\newcommand{\oof}[1]{\operatorname{O}\!\left({#1}\right)}
\newcommand{\oofi}[1]{\operatorname{O}({#1})}
\newcommand{\omegaof}[1]{\operatorname{\Omega}\!\left({#1}\right)}
\newcommand{\omegaofi}[1]{\operatorname{\Omega}({#1})}
\newcommand{\thetaof}[1]{\operatorname{\Theta}\!\left({#1}\right)}
\newcommand{\thetaofi}[1]{\operatorname{\Theta}({#1})}

%functions
\newcommand{\supp}[2][\mathit{u}]{\funct{supp_{#1}}{#2}}
\newcommand{\cnf}[2][\mathit{u}]{\funct{conf_{#1}}{#2}}
\newcommand{\lift}[2][\mathit{u}]{\funct{lift_{#1}}{#2}}
\newcommand{\sig}[2][\Theta]{\funct{sig_{_{#1}}}{#2}}
\newcommand{\wid}[1]{\funct{wid}{#1}}

% Domains and elements
\newcommand{\elem}[1]{\textsf{#1}}
\newcommand{\ont}{\mathcal{O}}
\newcommand{\clss}{\mathcal{E}}
\newcommand{\rels}{\mathcal{R}}

\newcommand{\msps}{\mathsf{msp}}
\newcommand{\minip}{\msps\!^{-}}

\newcommand{\bikingcentral}{\phi_{_{16}}}
\newcommand{\baseballcentral}{\phi_{_{20}}}
\newcommand{\sportcentral}{\phi_{_{15}}}
\newcommand{\ballcentral}{\phi_{_{17}}}
\newcommand{\bikballcentral}{\phi_{_{18}}}

\newcommand{\cgroup}{{c-group~}}
\newcommand{\cgroups}{{c-groups~}}

\newcommand{\cgroupnospace}{{c-group}}
\newcommand{\cgroupsnospace}{{c-groups}}

\title{An Iterative MapReduce Algorithm for
Approximating Closed Itemsets Using Spark}
\setlength{\paperheight}{11in}
\setlength{\paperwidth}{8.5in}

\usepackage[pdfborder={0 0 0}, plainpages, pdfpagelabels=false, pdfstartview=FitH]{hyperref}


\newfont{\mycrnotice}{ptmr8t at 7pt}
\newfont{\myconfname}{ptmri8t at 7pt}
\let\crnotice\mycrnotice%
\let\confname\myconfname%

\clubpenalty=10000
\widowpenalty = 10000

\begin{document}

\CopyrightYear{2016} 
\setcopyright{acmcopyright}
\conferenceinfo{SIGMOD'16,}{June 26-July 01, 2016, San Francisco, CA, USA}
\isbn{978-1-4503-3531-7/16/06}\acmPrice{\$15.00}
\doi{http://dx.doi.org/10.1145/2882903.2882922}

\clubpenalty=10000 
\widowpenalty = 10000

\numberofauthors{2} 
\author{
\alignauthor
Eyal Altshuler\\
       \affaddr{Tel-Aviv University}\\
       \affaddr{Tel-Aviv, Israel}
       \email{eyalalts@post.tau.ac.il}
% 2nd. author
\alignauthor
Tova Milo\\
       \affaddr{Tel-Aviv University}\\
       \affaddr{Tel-Aviv, Israel}
       \email{milo@post.tau.ac.il}
}

\maketitle

\section{Preliminaries}

\subsection{Notations}
Let $I$ be a set of items. $I=\{i_1,i_2,\ldots\}$.
Let $T$ be a database which is a set of
transactions, $T=\{t_1,\ldots,t_n\}$.
Each transaction in $T$ is a finite set of items, i.e, for
every $i$, $t_i\subseteq I$. Define an itemset as a finite
set of items. If $I'$ is an itemset, then $I' \subseteq I$.
The support of an itemset $I'$ in $T$ is the number of
transactions in T that include all items in $I'$ - 
\begin{center}
$supp_T(I') = |\{ t \in T | I' \subseteq t \}|$
\end{center}
The \textit{closure} of an itemset is the largest itemset
with the same support. Assume $P$ is an itemset, and $Q$ is
the closure of $P$. We mark by \textit{clo} the closure
operator- 
\begin{center}
$Q=clo(P)$
\end{center}
By its definition, we know that $supp(Q)=supp(P)$. In our
settings, consider the database is huge and therefore its
tuples can't fit in one machine main memory. Therefore, we
assume that transactions are loaded into a set of $k$
machines. As there are $n$ transactions, we assume that they
are loaded equally. Namely, $\frac{n}{k}$ transactions are
loaded to each machine. Given a database $T$ and an itemset
$P$, we want to compute the closure of $P$ in $T$. 

\begin{example}
\label{example:basic}
Assume the database $T$ is a list of supermarket receipts, 
each represents a single buyer. Assume the following
receipts (itemsets):
\begin{align*}
P_1 &= \{milk, coke\} \\
P_2 &= \{bread, butter, water\} \\
P_3 &= \{bread, butter, water, pasta\} \\
\end{align*}
If, for instance, there are 20 receipts containing \{bread,
butter, water\} then $supp_T(P_2) = 20$. Assume also that
$supp_T(P_3) = 20$, and there is no itemset $P_4$ such that
$P_3 \subset P_4$ and $supp(P_4) = 20$, then $clo(P_2) =
P_3$, as $P_3$ is the itemset with largest cardinality
having the same suppport in $T$ as $P_2$.
\end{example}

In general, we want to compute all closed itemsets having
support of at least $M$. From now on, we mark the desired threshold of an itemset by $M$.

We now destribe how to compute the closure of a given
itemset. For computing the closure of an itemset $P$, we
first need to project the entire database $T$ only on these
transactions in which $P$'s items occur. Then, after having 
this projection, we can simply compute the $closure$ operator by
instersecting all of its transactions. It is easy to prove
that this interscrtion will be exactly $P$'s closure, as it
will contain $P$, and any other item that is not in $P$ can
be added to it for obtaining its closure. As intersection is
a distributive opeartion, it can be easily applied in
our distributed settings of $k$ machines. Namely, all
machines are given $P$, and project their data on it. Then, they intersect all
projected transactions and get their result. These
results are partial as they were computed only
locally on each machine. These partial results are then sent
to a single machine, that intersects all results again, to
get the final result, which is  the closure of $P$.

As a final note, we consider the case of computing an
itemset closure when having only a sample of the data. When 
projecting a sample of the data on $P$, we never miss any of
the closure items in the result. the closure elements will 
appear if we intersect all the transactions in the projection. 
However, we might have some false positives. Namely, these 
are items that might appear in all of $P$'s occurences in 
the sample, but they badly represent the result when all data is tested.
Therefore, when applying the naive algorithm for computing
the closure of an itemset $P$, we get an itemset $Q'$ such
that $Q \subseteq Q'$ and $Q'$ might contain some elements
that are not part of the closure.

\begin{example}
Consider the database and itemsets presented in example
\ref{example:basic}. Consider a sub-sample $T$, having only
some of the receipts from $T$. Mark this sample by $S$.
Recall the itemset - $P_2 &= \{bread, butter,
water\}$. Assume we want to compute $clo(P_2)$ in $S$, or
simply $supp_S(P_2)$. Note that two cases are possible, depending 
on the support of $P_2$ in $S$. First case is that $supp_S(P_2)
= 20$, as happens in $T$. In this case, $P_3$ remains
the itemset with largest cardinality having the same support as $P_2$.
The second case is, for instance, when $supp_S(P_2) = 15$. 
Consider an itemset $P_4 = \{bread, butter,
water, pasta, rice\}$ such that $supp_T(P_4) = 15$. If
$supp_S(P_4) = supp_T(P_4) = 15$, then we get that
$clo_S(P_2) = P_4$. In this case, $P_4$ is having the same
support as $P_2$ in $S$. As $P_3 \subset P_4$, $P_4$ is an
itemset with higher cardinality than $P_3$ having the same
support in $S$ as $P_2$. Note that in such a sample, there
can be another itemset, $P_5$, such that $supp_S(P_4) =
supp_S(P_5)$ and $P_5$ contains all items of $P_2$. In such
case $P_5$ could also be considered the closure of $P_2$.
Having more than one candidate for the $closure$ operator,
we pick one randomly.
\end{example}

\subsection{LCM Algorithm}
We briefly explain the $LCM$ Algorithm \cite{uno2003lcm}, as
a baseline algorithm. $LCM$ builds a graph of closed
itemsets. It provides a technique for building the graph of all closed itemset.
Using an \textit{expansion} operator, it gives a way of
scanning a minimal amount of candidate itemsets. For every itemset, its
closure is computed, and then the closed itemset is
expanded. The resulting graph represents all closed itemsets
with frequency of at least $M$. Note that the tree is a
very compact way of representing the set of all
frequent itemsets. Given an itemset $i$, deciding whether
its frequency is at least is $M$ is done by scanning the graph
bottom up and finding a node identied with a set that
contains $i$. In case $i$ is frequent, its frequency will
be determined by finding the node that represents its
closure, which is the node with minimal cardinality 
(smallest number of items) containing $i$. 

 The LCM algorithm buids a graph of frequent closed itemsets
 levels. In every level, it starts by creating a set of $candidates$ itemsets. These are
frequent itemsets that have not yet tested by the algorithm.
Then, for every itemset in $candidates$, its
closure is computed. Let us mark the resulting set by
\textit{current-closed-itemsets}. The itemsets in
\textit{current-closed-itemsets} are then added to the graph
as a new level. In every insertion of a new level to the graph, all itemsets in 
the new level are connected with edges to 
itemsets in the previously inserted level.
Edges stand for the $subset$ condition. Namely, an edge
$(I_{old}, I_{new})$ means that $I_{old} \subset I_{new}$.
The algorithm works iteratively and explores a new set of
candidates in every iteration. It terminates when it
computes $candidates$ as an empty set. Then, the resulting
graph is returned.
%\begin{figure}
%\centering
%\hspace{-1mm}
%\includegraphics{figures/cube-lattice.pdf}
%\caption{Cube lattice} \label{fig:lattice}
%\end{figure}

\subsection{MapReduce Settings}
We specify here the MapReduce cluster settings that will be
used in the rest of the paper. Consider a database $T$ with
$n$ transactions of itemsets. Working in a distributed environment, 
suppose we have $k$ machines, and assume each one can run a single map
or a single reduce function in every map or reduce phase,
respectively. We assume that the $n$ transactions of the
input are equally loaded to the machines at the beginning of the algorithm.
We additionally assume that the machines have a main memory
that is in the order of their input size. In addition, 
we assume that all machines share a distributed file system,
in which $T$ is read from and to which the output closed
frequent itemsets will be written.

%\section{Related Work}
\label{sec:related}

\begin{itemize}
  \item Frequent Itemset Mining
  \item Closed frequent itemset mining
  \item MapReduce and Spark
  \item algorithms on big data using sampling
\end{itemize}

\section{Naive approach - distribuing the LCM algorithm}
\label{sec:alg}
In this section we explain how to parallelize the LCM
algorithm. We mark the parallelized version of LCM by
\textit{parallelized-lcm}. First we give an overview of the
distribution process, and then we express this parallelized
version as an iterative MapReduce algorithm. We end this
section by discussing some possible ways to optimize
performance of \textit{parallelized-lcm}. The optimizations
are explained intuitively, and used in Section \ref{sec:alg-sampling} as a
motivation to have the use of samping.

\subsection{Distributed LCM Primitives}
In a high level overview, \textit{parallelized-lcm} is
composed of a single master machine that iteratively runs
MapReduce rounds in the other machines and collects their
intermediate outputs. After completing all required
MapReduce rounds, the master machine return the tree of
closed frequent itemsets. The algorithm starts by elements
counting, that can be easily applied using MapReduce and is
expplained in $parallel-count$.
Then, only frequent elements are filtered, as they are the only
candidates element composing a frequent itemset. For
every frequent element, its closure is computed in parallel
by $parallel-closure$. The result is added as a closed
itemsets to the resulting tree. Having the new layer of all
candidates in the tree, the algorithm tries to expand and
find new candidates by applying $parallel-expand$ to every
closed itemset in the top level of the tree. The frequency
of every candidate is also computed using MapReduce, and the
candidates are only those itemsets having frequency larger
than $M$. We briefly describe how to
parallelize $parallel-count$, $parallel-closure$ and
$parallel-expand$ using MapReduce, in a way which is
straight forward. We then take into consideration our
algorithm settings for imrpoving them.
 
%\begin{algorithm}[t]
% \SetAlgoLined\DontPrintSemicolon
% // k mappers, Every mapper holds a subset of transactions
% \;
% \SetKwProg{myalg}{}{}{}
% \myalg{\textbf{Map(transaction)}}{
% \For{$item \in \textit{transaction}$}{
%    emit (item, 1)
% }
% }{}\;
% // k reducers, every reducer knows the item frequency
% threshold M
% \;
% \SetKwProg{myalg}{}{}{}
% \myalg{\textbf{Reduce(item, occurences)}}{
% \For{$value \in \textit{occurences}$}{
% 		item-count += value
% }
% \If{item-count > $M}{
% 		return (item, item-count)
% }{}\;
% }
% \caption{Frequent Items Count MapReduce algorithm}
% \label{alg:parallel-count}
% \end{algorithm}

MapReduce implementation of $parallel-count$ is similar to
the word-count MapReduce algorithm [?] where every item is
considered as a word. The MapReduce implementation of
$parallel-closure$ is as follows. In the Map phase, every
mapper filters its transactions acordding to its input
itemset, and then applies intersection on all transactions. 
The resulting itemsets, which are actually the
closure computed locally at this mapper, are sent to a
single reducer. Only a single reducer is required to
complete the aggregation and compute the final closure. The single
reducer simply intersects all of its input, having the
closure as a result.

% \begin{algorithm}[t]
%  \SetAlgoLined\DontPrintSemicolon
%  // k mappers, Every mapper holds a subset of transactions
%  and an itemset $I$
%  \;
%  \SetKwProg{myalg}{}{}{}
%  \myalg{\textbf{Map(transactions)}}{
%  \For{$item \in \textit{transactions}$}{
%     filtered-transactions =
%     filter-transactions-containing(transactions, I);  \\
%     partial-intersection =
%     Intersection(filtered-transactions); \\
%     emit (I, partial-intersection) 
%  }
%  }{}\;
%  // 1 reducer
%  \;
%  \SetKwProg{myalg}{}{}{}
%  \myalg{\textbf{Reduce(I, List<partial-intersection> l)}}{
%  	emit (I, Intersection(l))
%  {}\;
%  }
%  \caption{Parallel-Closure MapReduce algorithm}
%  \label{alg:parallel-closure}
%  \end{algorithm}

For a MapReduce implementation of $parallel-expand$, only a
map phase is needed. Namely, no aggregation is needed in the
reduce phase and each mapper can simply return all of its
outputs. The work every mapper does is going over its
itemsets, and for every itemset of size $s$, emiting all
possible itemsets of size $s+1$ by adding them a single item. 
For doing this, we assume that the set of all possible items 
is known to every mapper in advance. We note that this requirement might
consume some mapper memory, but this is a reasonable assumption in pracatical 
MapReduce settings.

% \begin{algorithm}[t]
% \SetAlgoLined\DontPrintSemicolon
% // k mappers, Every mapper holds an itemset $I$,
% and knows in memory the set of all items in the data,
% marked by $I_all$
% \;
% \SetKwProg{myalg}{}{}{}
% \myalg{\textbf{Map(I)}}{
% elements-to-add = $I_all$ \ $I$
% \For{$item \in \textit{elements-to-add}$}{
% emit (I, $I \cup item$)
% }
% }{}\;
% \caption{Parallel-expand MapReduce algorithm}
% \label{alg:parallel-expand}
% \end{algorithm}

\subsection{Computing a Batch of Closures}
We now suggest an optimization to $parallelized-lcm$,
regarding the computation of itemsets closure. When the
master machine decides to compute a closure for an itemset
$I$, it applies $parallelized-closure$. By doing this, as
described in Algorithm ??, all mappers project their
transactions on $I$ and intersecting for having the local
closure. All results are sent to a reducer to complete the
intersection. We suggest $batched-parallelized-closure$ in
Algorithm ??. In this algorithm, every mapper gets a
batch of itesmsets as an input. Assume a batch
${I_1,\ldots,I_t}$. Then, every mapper applies $t$
projection on its transaction, one for each itemset. Every
projection is then intersected and sent to a reducer. Reduce
function remains the same. Compared to
$parallelized-closure$, this mapper has more work to do.
However, used in $parallelized-lcm$, we suggest that the
resulting tree will be composed in layers. Namely, after
doing $parallel-expand$ and having the set of candidates to
compute closure for, all candidates are sent together to
$batched-parallelized-closure$. We call the resulting
algorithm $batched-parallelized-lcm$ and describe its pseudo
code in Algorithm ??. By computing the closures in batches,
we save the IO time of loading the data to mappers for every
itemset, and using data loaded for a batch of itemsets
together.

\subsection{The Speed-Accuracy tradeoff}
In $batched-parallelized-lcm$, there are many MapReduce jobs
that occur frequently in the algorithm and required the work
of all mappers and reducers. We note that if we ask only few
mappers to work for some input, it is equivalent to applying
the MapReduce job only on a sample of the input. For
instance, having the transactions equally divided between
$n$ mappers, applying the MapReduce job only on
$\frac{n}{2}$ mappers is equivalent to running the algorithm on half of
the transactions. The operations of elements counting and
closure computation can be done on a sample of the
data, or in other words only on a subset of mappers,
resulting in approximated results. However, as we
have $n$ mappers, we can run the approximated jobs
together. In the current example, We can run in
parallel two MapReduce jobs, each on $\frac{n}{2}$
mappers. We lose the accuracy of the result, but gain
a significant factor of running time. In section
\ref{sec:alg-sampling}, we suggest a formal
definition for the approximated computation, and
determine the best balance for having a fast
and well approximated computation.
\section{Speeding up Distributed LCM using sampling}
\label{sec:alg-sampling}
In this section we present an approximated version of
$batched-parallelized-lcm$ presented in Section
\ref{sec:alg}. We use sampling in two main components 
of the algorithm. The technique of sampling enables 
the algorithm to dramatically reduce the amount of 
processed data. As we use sampling for itemsets counting
and closure computation, the algorithm gives an
approximation of all closed frequent itemsets. We give
theoritical analysis of the approximation. Then desbribe the 
approximated version of the algorithm. We end this section by theoretically analyzing 
the efficiency of using sampling, in two aspects. First is 
running time, achieved due to the reduced amount of data 
to be processed. Second is the performance of approximation. 
We show that it is very accurate and catches the frequent 
closed itemsets very well.

\subsection{Itemset Frequency Approximation}
Algorithm \ref{alg:batched-parallel-lcm} uses MapReduce
to compute the frequency of a given itemset in the
data. Such MapReduce job was desctibed in Algorithm
\ref{alg:parallel-count}. We use to following lemma to
show that a random sample of the data can be used
to estimate the frequency of an itemset-

\begin{lemma}
If we take a sample of size $S$, the frequenct of an Itemset
$I$ is well approximated.
\end{lemma}

\begin{proof}
Using Chernoff bound
\end{proof}

Note that by these result, we can use a very small sample
for estimating the frequency of every candidate itemset in a
batch. In practical cases, the number of transactions in a
single mapper are extremely larger than those required to
ensure an almost 99.9\% approximation. We therefore use a
single MapReduce job having only a MapPhase, which is
equivalleny to letting every itemset to be estimated locally
in every mapper. Note that we assume that data is randomly
shufflled among the machines.

\subsection{Presenting $(1-\alpha)$ closure operator}
We show that a sample can be used to estimate the closure of
a given itemset. Note that computing the closure of an
itemset on a sample, can result in an itemset that contains
all items of the closure, but might contain some more.
For closure estimation, we formally define the
approximation of the closure. Let $0 < \alpha < 1$. We first define the $(1-\alpha)-closure$.

Consider an itemset $P$, its closure $Q$ and $\alpha$. An
itemset $Q'$ is a $(1-\alpha)-closure$ of $P$ if $Q
\subseteq Q'$ and $supp(Q')\geq (1-\alpha)supp(P)$. Note
that according to its definition-
\begin{center}
$supp(P)\geq supp(Q')$
\end{center}
Additionally, note that there might be more than one
$(1-\alpha)-closure$ for a given itemset $P$ and a given
$\alpha$. This is in contrary to the uniqueness of $Q$,
which is the largest existing itemset that contain $P$ and
has the same support.
If we pick a small enough $\alpha$, then the
$(1-\alpha)-closure$ of $P$ will be the real closure. We
show that using sampling, we can pick a relatively small
sample size for computing a $(1-\alpha)-closure$ of $P$, for
practically very small sizes of $\alpha$.

We use the next two lemmas for adding a preprocessing
MapReduce at the beginning that allows us to have an very
good approximation of the graph, but works on a very small
portion of the data. Thus, we are able to massively scale
the algorithm. The sample can be done using MapReduce too,
for having it fast. The entire algorithm can be implemented
in frameworks such as Spark, that allows iterative MapReduce algorithms to be
done only on main memory instead of reading and writing on
disk between phases.

Consider a sample of size $s$. We use the following two
lemmas:
\begin{lemma}
\eyal{FORMAL DEFINITION}
If an itemset is having a frequency $M$, then it
appears at least $x$ in the sample with very high probability.
\end{lemma}
\begin{proof}
\eyal{complete proof}
\end{proof}
\begin{lemma} 
\eyal{FORMAL DEFINITION}
If an itemset is having a frequency
$M$, then its closure in the sample is $(1-\alpha)$-closure with very
 high probability.
\end{lemma}
\begin{proof}
\eyal{complete proof}
\end{proof}

\subsection{Closure Approximation}
Let us now show, given $\alpha$, how to approximate the
$(1-\alpha)-closure$ of an itemset $P$ using sampling.
Consider an itemset $P$ and its closure, $Q'$. Then, given a
sample of $S$ transactions, denote by $X$ the number of
occurences of $Q'$. Then, $X$ is a random variable and let
us denote by $X_i$ the indicator random variable that marks
whehter $Q'$ is a subset of the $i$-th transaction in the
sample. Then -

\[
 X_i =
  \begin{cases}
   1       & \text{Q' is the i'th transaction}\\
   0       & \text{otherwise}
  \end{cases}
\]

In addition - 

\[ {E({X)}} =
{E(\sum\limits_{i=1}^{S}{X_i})}
\]

According to its definition, we know that-

\[ {E({X_i)}} \geq
(1-\alpha)*\frac{\left\vert{P}\right\vert}{n}
\]

And therefore we know the lower bound for the expectation of
$X$ - 

\[ {E({X)}} \geq
(1-\alpha)*\frac{S * \left\vert{P}\right\vert}{n}
\]

We now use Chernoff inequality to bound the probability
that $X$ differs from its expectation by a factor of
$\delta$. Then - 
\begin{equation*}
\begin{split}
\Pr(X < (1-\delta)*E(X)) & < \\
\Pr(X < (1-\alpha)*\frac{S * \left\vert{P}\right\vert}{n}) < \\
exp^{-(1-\alpha)*\frac{S *\left\vert{P}\right\vert}{n}\frac{\delta^2}{2}} \\
\end{split}
\end{equation*}

Note that this error is monotoically decreasing in the size
of $S$. Namely, the larger we take $S$ to be, the smaller
error we have. If for instance, we want to bound the error
of missing $Q'$ by $\epsilon$, then -
\begin{equation}
exp^{-(1-\alpha)*\frac{S
*\left\vert{P}\right\vert}{n}\frac{\delta^2}{2}} <
\epsilon\\
\end{equation}

and we should take $S$ to be of size - 
\begin{equation}
S >
ln(\frac{1}{\epsilon})*\frac{2n}{(1-\alpha)\delta^2\left\vert{P}\right\vert}
\end{equation}

To sum up, if we take $S$ to be at least of this size, then
we bound the error of missing $Q'$. In other words, if we
take $S$ of such size, we get, with probability of at least
$1-\epsilon$, a set that includes the closure of $P$ and
exists in more than $(1-\alpha)$ where $P$ exists.

\subsection{Iterative MapReduce Implementation}
We end this section by describing some minor changes that
need to be made to $batched-parallelized-lcm$.
We describe the new algorithm, called $approximated-lcm$.
The pseudo-code of $approximated-lcm$ can is shown in
Algorithm \ref{alg:approximated-lcm}.

In a high level overview, the algorithm is running by a
master machine that manages its logic, and whenever a
MapReduce step is needed, data is loaded in parallel into
the main memory of all slaves machines, that run MapReduce
on it and send the result back to the master. The master is
then able to decide about the next step. The master
machine asks the slaves of running MapReduce steps of two
types. First type is the itemset approximation, and the
second is batched itemset closure approximation. The
algorithm terminates after the master completes asking the 
slaves for running MapReduce rounds. The algorithm output 
is then written by the master.

\begin{algorithm}[t]
 \DontPrintSemicolon
 //A master machine and k slaves machines running
 MapReduce\;
 \SetKwProg{myalg}{}{}{}
 \myalg{\textbf{APPROXIMATED-LCM}}{
 	sample =  MapReduceSample(); \\
 	load-to-slaves(sample); \\
 	items-counts = parallel-count(items); \\
 	candidates = filter(item-counts, M); \\
 	closed-tree = Initialize-closed-tree(); \\
 	\While{candidates}{
 		closed-itemsets = parallel-closure(candidates); \\
 		add-level-to-tree(closed-tree, closed-itemsets); \\
 		next-level = parallel-expand(closed-itemsets); \\
 		candidates = filter(next-level, M); \\
 	}
 }{}\;
 \caption{Approximated LCM algorithm}
 \label{alg:approximated-lcm}
\end{algorithm}

The algorithm starts by randomly sampling the data, with a 
sample size $s$ determined by Lemmas ?? and ??. The goal is 
to have a good approximations of both items frequency and 
itemsets closure. Data sampling is also done using a MapReduce 
iteration in which every mapper samples $\frac{1}{sk}$ and 
therefore the expectation of the sample size is $s$. All 
mappers send their to a single reducer. In practical settings, 
the sample is small enough to fit in a single machine memory, 
and therefore a single reducer is picked.

After having the sample, A first step of type itemset
frequency approximation is applied on it. This step is an
example itemset approximation, and it is done using MapReduce in which the
sample is split accorss all machines and they simply do
itemset counting. After completing MapReduce singletons
itemsets counting, the results are sent to the master and
only items having support of at least $M$ are filtered as
candidates of building frequent itemsets in general. The
master is then using a batched closure computation step on
the sample to find the closure of every singleton itemset.
Note that by Lemma ??, the master is able to determine the
required number of machines for every closure computation.
To simplify our description of the algorithm, we assume that
a single machine is having a large enough amount of data for
having a goog approximation for a singleton itemsets.

Note that this is a reasonable assumption in practical
settings, in case of having a large frequency of an itemset
and in case of a lower thresold. In addition, as the
frequency of scanned itemsets is only getting smaller, if we
assume that a single machine is required for having a good
approximation of the closure of a singleton itemsets, then a
single machine is required for every candidate itemsets
scanned by the algorithm.

After the master is having a batch of a filtered
singleton itemsets with support larger than $M$, it has
to assign every mapper the itemsets to work on. In our
case, it has to take the batch of itemsets and equally
divide them between mappers. In the general case, in
which the number of required machines can be between one
and $t$, an itemset closure might be computed using more
than one machine. Deciding the optimal assignment of
itemsets to mappers reduces to the problem of resource
allocation, which is known to be NP-hard. In our work we
use hueristic that approximates the optimal allocation. We
adopt the algorithm from ??.

After computing the closure approximation, the master has
the first level of closed frequent itemsets in the closed
frequent itemsets tree. The next step is applying the
expand operation, which was described in ??. The resulting
itemsets of the expand operation are the candidates on which
a closure approximation will be computed in the next level.
Before applying this computation, the master agains using
the MapReduce itemset approximation step for filtering only
the frequent candidates. Then they are assigned to mappers
for computing the closure approximation step, as before.
Results are appended to the resulting tree. The master then
iteratively scans all candidates, until one of the two
possible situations. First is that all candidates itemsets
are not considered as frequent after approximation. Second
is that the computed closed itemsets is containing all
frequent itetms, and there cannot be expanded, resulting in
an empty set of candidates itemsets by the expand operator.

%\bibliographystyle{abbrv}
%\bibliography{bib}
%\input{appendix}

\end{document}
