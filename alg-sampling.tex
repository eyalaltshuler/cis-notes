\section{Speeding up Distributed LCM using sampling}
\label{sec:alg-sampling}
In this section we present an approximated version of
$batched-parallelized-lcm$ presented in Section
\ref{sec:alg}. We use sampling in two main components 
of the algorithm. The technique of sampling enables 
the algorithm to dramatically reduce the amount of 
processed data. As we use sampling for itemsets counting
and closure computation, the algorithm gives an
approximation of all closed frequent itemsets. We give
theoritical analysis of the approximation. Then desbribe the 
approximated version of the algorithm. We end this section by theoretically analyzing 
the efficiency of using sampling, in two aspects. First is 
running time, achieved due to the reduced amount of data 
to be processed. Second is the performance of approximation. 
We show that it is very accurate and catches the frequent 
closed itemsets very well.

\subsection{Itemset Frequency Approximation}
Algorithm \ref{alg:batched-parallel-lcm} uses MapReduce
to compute the frequency of a given itemset in the
data. Such MapReduce job was desctibed in Algorithm
\ref{alg:parallel-count}. We use to following lemma to
show that a random sample of the data can be used
to estimate the frequency of an itemset-

\begin{lemma}
If we take a sample of size $S$, the frequenct of an Itemset
$I$ is well approximated.
\end{lemma}

\begin{proof}
Using Chernoff bound
\end{proof}

Note that by these result, we can use a very small sample
for estimating the frequency of every candidate itemset in a
batch. In practical cases, the number of transactions in a
single mapper are extremely larger than those required to
ensure an almost 99.9\% approximation. We therefore use a
single MapReduce job having only a MapPhase, which is
equivalleny to letting every itemset to be estimated locally
in every mapper. Note that we assume that data is randomly
shufflled among the machines.

\subsection{Presenting $(1-\alpha)$ closure operator}
We show that a sample can be used to estimate the closure of
a given itemset. Note that computing the closure of an
itemset on a sample, can result in an itemset that contains
all items of the closure, but might contain some more.
For closure estimation, we formally define the
approximation of the closure. Let $0 < \alpha < 1$. We first define the $(1-\alpha)-closure$.

Consider an itemset $P$, its closure $Q$ and $\alpha$. An
itemset $Q'$ is a $(1-\alpha)-closure$ of $P$ if $Q
\subseteq Q'$ and $supp(Q')\geq (1-\alpha)supp(P)$. Note
that according to its definition-
\begin{center}
$supp(P)\geq supp(Q')$
\end{center}
Additionally, note that there might be more than one
$(1-\alpha)-closure$ for a given itemset $P$ and a given
$\alpha$. This is in contrary to the uniqueness of $Q$,
which is the largest existing itemset that contain $P$ and
has the same support.
If we pick a small enough $\alpha$, then the
$(1-\alpha)-closure$ of $P$ will be the real closure. We
show that using sampling, we can pick a relatively small
sample size for computing a $(1-\alpha)-closure$ of $P$, for
practically very small sizes of $\alpha$.

We use the next two lemmas for adding a preprocessing
MapReduce at the beginning that allows us to have an very
good approximation of the graph, but works on a very small
portion of the data. Thus, we are able to massively scale
the algorithm. The sample can be done using MapReduce too,
for having it fast. The entire algorithm can be implemented
in frameworks such as Spark, that allows iterative MapReduce algorithms to be
done only on main memory instead of reading and writing on
disk between phases.

Consider a sample of size $s$. We use the following two
lemmas:
\begin{lemma}
\eyal{FORMAL DEFINITION}
If an itemset is having a frequency $M$, then it
appears at least $x$ in the sample with very high probability.
\end{lemma}
\begin{proof}
\eyal{complete proof}
\end{proof}
\begin{lemma} 
\eyal{FORMAL DEFINITION}
If an itemset is having a frequency
$M$, then its closure in the sample is $(1-\alpha)$-closure with very
 high probability.
\end{lemma}
\begin{proof}
\eyal{complete proof}
\end{proof}

\subsection{Closure Approximation}
Let us now show, given $\alpha$, how to approximate the
$(1-\alpha)-closure$ of an itemset $P$ using sampling.
Consider an itemset $P$ and its closure, $Q'$. Then, given a
sample of $S$ transactions, denote by $X$ the number of
occurences of $Q'$. Then, $X$ is a random variable and let
us denote by $X_i$ the indicator random variable that marks
whehter $Q'$ is a subset of the $i$-th transaction in the
sample. Then -

\[
 X_i =
  \begin{cases}
   1       & \text{Q' is the i'th transaction}\\
   0       & \text{otherwise}
  \end{cases}
\]

In addition - 

\[ {E({X)}} =
{E(\sum\limits_{i=1}^{S}{X_i})}
\]

According to its definition, we know that-

\[ {E({X_i)}} \geq
(1-\alpha)*\frac{\left\vert{P}\right\vert}{n}
\]

And therefore we know the lower bound for the expectation of
$X$ - 

\[ {E({X)}} \geq
(1-\alpha)*\frac{S * \left\vert{P}\right\vert}{n}
\]

We now use Chernoff inequality to bound the probability
that $X$ differs from its expectation by a factor of
$\delta$. Then - 
\begin{equation*}
\begin{split}
\Pr(X < (1-\delta)*E(X)) & < \\
\Pr(X < (1-\alpha)*\frac{S * \left\vert{P}\right\vert}{n}) < \\
exp^{-(1-\alpha)*\frac{S *\left\vert{P}\right\vert}{n}\frac{\delta^2}{2}} \\
\end{split}
\end{equation*}

Note that this error is monotoically decreasing in the size
of $S$. Namely, the larger we take $S$ to be, the smaller
error we have. If for instance, we want to bound the error
of missing $Q'$ by $\epsilon$, then -
\begin{equation}
exp^{-(1-\alpha)*\frac{S
*\left\vert{P}\right\vert}{n}\frac{\delta^2}{2}} <
\epsilon\\
\end{equation}

and we should take $S$ to be of size - 
\begin{equation}
S >
ln(\frac{1}{\epsilon})*\frac{2n}{(1-\alpha)\delta^2\left\vert{P}\right\vert}
\end{equation}

To sum up, if we take $S$ to be at least of this size, then
we bound the error of missing $Q'$. In other words, if we
take $S$ of such size, we get, with probability of at least
$1-\epsilon$, a set that includes the closure of $P$ and
exists in more than $(1-\alpha)$ where $P$ exists.

\subsection{Iterative MapReduce Implementation}
We end this section by describing some minor changes that
need to be made to $batched-parallelized-lcm$.
We describe the new algorithm, called $approximated-lcm$.
The pseudo-code of $approximated-lcm$ can is shown in
Algorithm \ref{alg:approximated-lcm}.

In a high level overview, the algorithm is running by a
master machine that manages its logic, and whenever a
MapReduce step is needed, data is loaded in parallel into
the main memory of all slaves machines, that run MapReduce
on it and send the result back to the master. The master is
then able to decide about the next step. The master
machine asks the slaves of running MapReduce steps of two
types. First type is the itemset approximation, and the
second is batched itemset closure approximation. The
algorithm terminates after the master completes asking the 
slaves for running MapReduce rounds. The algorithm output 
is then written by the master.

\begin{algorithm}[t]
 \DontPrintSemicolon
 //A master machine and k slaves machines running
 MapReduce\;
 \SetKwProg{myalg}{}{}{}
 \myalg{\textbf{APPROXIMATED-LCM}}{
 	sample =  MapReduceSample(); \\
 	load-to-slaves(sample); \\
 	items-counts = parallel-count(items); \\
 	candidates = filter(item-counts, M); \\
 	closed-tree = Initialize-closed-tree(); \\
 	\While{candidates}{
 		closed-itemsets = parallel-closure(candidates); \\
 		add-level-to-tree(closed-tree, closed-itemsets); \\
 		next-level = parallel-expand(closed-itemsets); \\
 		candidates = filter(next-level, M); \\
 	}
 }{}\;
 \caption{Approximated LCM algorithm}
 \label{alg:approximated-lcm}
\end{algorithm}

The algorithm starts by randomly sampling the data, with a 
sample size $s$ determined by Lemmas ?? and ??. The goal is 
to have a good approximations of both items frequency and 
itemsets closure. Data sampling is also done using a MapReduce 
iteration in which every mapper samples $\frac{1}{sk}$ and 
therefore the expectation of the sample size is $s$. All 
mappers send their to a single reducer. In practical settings, 
the sample is small enough to fit in a single machine memory, 
and therefore a single reducer is picked.

After having the sample, A first step of type itemset
frequency approximation is applied on it. This step is an
example itemset approximation, and it is done using MapReduce in which the
sample is split accorss all machines and they simply do
itemset counting. After completing MapReduce singletons
itemsets counting, the results are sent to the master and
only items having support of at least $M$ are filtered as
candidates of building frequent itemsets in general. The
master is then using a batched closure computation step on
the sample to find the closure of every singleton itemset.
Note that by Lemma ??, the master is able to determine the
required number of machines for every closure computation.
To simplify our description of the algorithm, we assume that
a single machine is having a large enough amount of data for
having a goog approximation for a singleton itemsets.

Note that this is a reasonable assumption in practical
settings, in case of having a large frequency of an itemset
and in case of a lower thresold. In addition, as the
frequency of scanned itemsets is only getting smaller, if we
assume that a single machine is required for having a good
approximation of the closure of a singleton itemsets, then a
single machine is required for every candidate itemsets
scanned by the algorithm.

After the master is having a batch of a filtered
singleton itemsets with support larger than $M$, it has
to assign every mapper the itemsets to work on. In our
case, it has to take the batch of itemsets and equally
divide them between mappers. In the general case, in
which the number of required machines can be between one
and $t$, an itemset closure might be computed using more
than one machine. Deciding the optimal assignment of
itemsets to mappers reduces to the problem of resource
allocation, which is known to be NP-hard. In our work we
use hueristic that approximates the optimal allocation. We
adopt the algorithm from ??.

After computing the closure approximation, the master has
the first level of closed frequent itemsets in the closed
frequent itemsets tree. The next step is applying the
expand operation, which was described in ??. The resulting
itemsets of the expand operation are the candidates on which
a closure approximation will be computed in the next level.
Before applying this computation, the master agains using
the MapReduce itemset approximation step for filtering only
the frequent candidates. Then they are assigned to mappers
for computing the closure approximation step, as before.
Results are appended to the resulting tree. The master then
iteratively scans all candidates, until one of the two
possible situations. First is that all candidates itemsets
are not considered as frequent after approximation. Second
is that the computed closed itemsets is containing all
frequent itetms, and there cannot be expanded, resulting in
an empty set of candidates itemsets by the expand operator.
